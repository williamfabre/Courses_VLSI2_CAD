\documentclass[11pt,a4paper,sans]{report}
\input{header.sty}

\begin{document}
\begin{titlepage}

	\center % Center everything on the page

	%----------------------------------------------------------------------------------------
	%	HEADING SECTIONS
	%----------------------------------------------------------------------------------------

	\textsc{\LARGE University Pierre et Marie Curie}\\[1.5cm] % Name of your university/college
	\textsc{\Large project MOCCA}\\[0.5cm] % Major heading such as course name

	%----------------------------------------------------------------------------------------
	%	TITLE SECTION
	%----------------------------------------------------------------------------------------
	\vfill
	\HRule \\[0.4cm]
	{ \huge \bfseries Report MOCCA : Synthesys, place and route of a given MIPS architecture with CADENCE toolchain}\\[0.4cm] 
	\HRule \\[1.5cm]
	\vfill
	%----------------------------------------------------------------------------------------
	%	AUTHOR SECTION
	%----------------------------------------------------------------------------------------

	\begin{minipage}{0.4\textwidth}
		\begin{flushleft} \large
			\emph{Authors:}\\
			% Ordre alphabetique sur les noms
			\textsc{Youcef Sekouri, William Fabre} 
		\end{flushleft}
	\end{minipage}
	~
	\begin{minipage}{0.4\textwidth}
		\begin{flushright} \large
			\emph{Teachers :} \\
			Mr \textsc{Matthieu Tuna},\textsc{Pirouz Bazarghan-Sabet}
		\end{flushright}
	\end{minipage}\\[2cm]

	%----------------------------------------------------------------------------------------
	%	DATE SECTION
	%----------------------------------------------------------------------------------------

	{\large Year 2019-2020}\\[2cm] % Date, change the \today to a set date if you want to be precise

\end{titlepage}

\newpage
\tableofcontents
\vspace*{3cm}
\begingroup\let\clearpage\relax

	\newpage
	\chapter{Introduction}

	%multilinecomments
	\begin{comment}
	\end{comment}

	% TODO quotation work use the biblio.bib to add references.
	TODOCHANGEHEREtest1234\cite{greenwade93}

	\newpage
	\chapter{Project Installation}
	% TODO tutoriel d'installation
	\newpage


	\newpage
	\bibliographystyle{IEEEtran}
	\bibliography{biblio}

	\newpage
	% \phantomsection
	\addcontentsline{toc}{chapter}{\listfigurename}
	\listoffigures



	\newpage
	\chapter{Annexe}
	\section{Glossary :}

\begin{itemize}
	\item EDA : Electronic design automation
	\item DFT : design for test %TODO
	\item CTS : Clock Tree Synthesys : %TODO
	\item STA : Static Timing Analysis  %TODO operates independently of characterization reading both a Verilog netlist and multiple timing libraries in Liberty format 

	\item Formal Verification :
	\item DRC : Design Rule Check %TODO
	\item LVS : Logical Vs Schematic %TODO extraction of circuit
	\item IC : Integrated Circuit
	\item PCB : Printed Circuit Board
	\item Plot : %TODO
	\item Pad : %TODO
	\item FP : Floot Planing %TODO (guide the tool)
	\item ATPG : Automatic Test Pattern generation
	\item SDC :

	\item PVT : process voltage temperature %TODO?
		%Lot to Lot
		%Wafer to Wafer
		%Dye to Dye
		%Corner-based analysis?

	% TODO
	\item Timing Constraints :
		Reg to Reg
		Input to Reg
		Reg to Output
		Input Delay
		Output Delay
		Input to output : Mealy path (Combinatory cannot be constraints) so you need to contraint input/output with absolut value


	\item Fanout : how much cells my signal can attack without being degraded

	\item Setup :
	\item Hold :

	\item .lib :Technology library source files

	\item library max (WC=Worst Case) : The timing in this kind of library are the longest (Max path/Max Data path). They are used for setup analysis. 
	\item library in (BC=Best Case) : The timing in the kind of library are the shortest (Min path/Min Data path). They are used for hold analysis.
	

	\item WNS : Worst negative slack : The longest path in the design
	\item TNS : Total Negative Slack : The Sum of all the longest path in the design
	\item WLM : Wire Load Model : Statistical Model that help having a coherent
		analysis of the circuit after synthesys (there is no wire yet).


	\item TA  : Timing Arc %TODO
		* Cell
			Cell Delay : propagation delay inside a cell

			Fall : inside cell Delay to transition from 1 to 0
			Rise : inside ell Delay to transition from 0 to 1

		* Transition
			fall : (a,b) in CELLS :
				Delay from cell a start going down to cell b that start going down
			rise: (a,b) in CELLS :
				Delay from cell a start going up to cell b that start going up

	\item Delay equation %TODO
	\item TC  : Timing Clean %TODO

	\item 
	\item 
	\item 
	\item 
	\item 
	\item 
	\item 
	\item 
	\item 
	\item 
	\item 
	\item 
	\item 
	\item 
	\item 
\end{itemize}



\end{document}

