\documentclass[11pt,a4paper,sans]{report}
%%%%%%%%%%%%%%%%%%%%%%%%%%%%%%%%%HEADER FROM ADRIAN SATIN, modified by william fabre %%%%%%%%%%%%%%%%%%%%%%%%%%%%%%%%%%%%%%
\usepackage[utf8]{inputenc} 
\usepackage{graphicx} % support the \includegraphics command and options
%\usepackage[frenchb]{babel}
\usepackage{tikz}
%*\usepackage{circuitikz}
\usetikzlibrary{circuits}
\graphicspath{{images/}}% chemin vers les images
\usepackage[parfill]{parskip} % Activate to begin paragraphs with an empty line rather than an indent
%%% PACKAGES
\usepackage{hyperref} % Gestion Hyperliens/url
% https://en.wikibooks.org/wiki/LaTeX/Hyperlinks
\usepackage{eurosym}
\usepackage{fancyhdr}
\usepackage{color}
\usepackage{svg}
% YOLO NICE CODE EXEMPLE merci axel
%\usepackage{minted}
%\usepackage{paralist} % very flexible & customisable lists (eg. enumerate/itemize, etc.)
%\usepackage{verbatim} % adds environment for commenting out blocks of text & for better verbatim
\usepackage{subfig} % make it possible to include more than one captioned figure/table in a single float
\usepackage{graphicx}
\usepackage{fancyhdr}
\usepackage{multicol}
\usepackage{listings}
\usepackage{amsmath,amsfonts,amsthm,amssymb}
\usepackage{pdfpages}
\usepackage{comment}
\usepackage{caption}
\definecolor{violet-logo}{RGB}{23,63,28}
%%% HEADERS & FOOTERS
\usepackage{fancyhdr} % This should be set AFTER setting up the page geometry
\pagestyle{plain} % options: empty , plain , fancy
\renewcommand{\headrulewidth}{1pt} % customise the layout...
\lfoot{}\cfoot{\thepage}\rfoot{}
%%% SECTION TITLE APPEARANCE
%*\usepackage{sectsty}
\allsectionsfont{\sffamily\mdseries\upshape} % (See the fntguide.pdf for font help)
% (This matches ConTeXt defaults)
%%% ToC (table of contents) APPEARANCE
\usepackage[nottoc,notlof,notlot]{tocbibind} % Put the bibliography in the ToC
%*\usepackage[titles,subfigure]{tocloft} % Alter the style of the Table of Contents
\renewcommand{\cftsecfont}{\rmfamily\mdseries\upshape}
\renewcommand{\cftsecpagefont}{\rmfamily\mdseries\upshape} % No bold!
\usepackage{geometry} % to change the page dimensions
\geometry{left=2cm, right=2cm, bottom= 1cm}
\geometry{a4paper} % or letterpaper (US) or a5paper or....
\fancyhead{}
%%% END Article customizations

%nice chapter
\usepackage{titlesec}
\titleformat{\chapter}[display]
{\normalfont\bfseries}{}{0pt}{\Large}
\titlespacing*{\chapter}{0pt}{-50pt}{40pt}

% nice section
\makeatletter
\def\@seccntformat#1{%
	\expandafter\ifx\csname c@#1\endcsname\c@section\else
\csname the#1\endcsname\quad
  \fi}
\makeatother

\usepackage{array,multirow,makecell}
\setcellgapes{1pt}
\makegapedcells
\newcommand{\HRule}{\rule{\linewidth}{0.5mm}} % Defines a new command for the horizontal lines, change thickness here
\fancyhf{} % sets both header and footer to nothing
\renewcommand{\headrulewidth}{0pt}
\addtolength{\topmargin}{-60pt}

%bibtex
\bibliographystyle{plain}

% defining my own style for code
%\usepackage{xcolor}
\usepackage[dvipsnames]{xcolor}
\definecolor{codegreen}{rgb}{0,0.6,0}
\definecolor{codegray}{rgb}{0.5,0.5,0.5}
\definecolor{codepurple}{rgb}{0.58,0,0.82}
\definecolor{backcolour}{rgb}{0.95,0.95,0.92}

\lstdefinestyle{mystyle}{
	%basicstyle=\fontsize\tiny\ttfamily %proper font size
	%basicstyle=\small, %or \small or \footnotesize etc.
	backgroundcolor=\color{backcolour},   
	commentstyle=\color{codegreen},
	keywordstyle=\color{magenta},
	numberstyle=\tiny\color{codegray},
	stringstyle=\color{codepurple},
	basicstyle=\ttfamily\footnotesize,
	breakatwhitespace=false,         
	breaklines=true,                 
	captionpos=t,                    
	keepspaces=true,                 
	numbers=left,                    
	numbersep=5pt,                  
	showspaces=false,                
	showstringspaces=false,
	showtabs=false,                  
	tabsize=2
}

\lstset{style=mystyle}


% my own command to add caption to figure code
\newcommand{\mylisting}[2][]{%
	\lstinputlisting[caption={\texttt{\detokenize{#2}}},#1]{#2}%
}



\begin{document}
\begin{titlepage}

	\center % Center everything on the page

	%----------------------------------------------------------------------------------------
	%	HEADING SECTIONS
	%----------------------------------------------------------------------------------------

	\textsc{\LARGE University Pierre et Marie Curie}\\[1.5cm] % Name of your university/college
	\textsc{\Large project MOCCA}\\[0.5cm] % Major heading such as course name

	%----------------------------------------------------------------------------------------
	%	TITLE SECTION
	%----------------------------------------------------------------------------------------
	\vfill
	\HRule \\[0.4cm]
	{ \huge \bfseries Report MOCCA : Synthesys, place and route of a given MIPS architecture with CADENCE toolchain}\\[0.4cm] 
	\HRule \\[1.5cm]
	\vfill
	%----------------------------------------------------------------------------------------
	%	AUTHOR SECTION
	%----------------------------------------------------------------------------------------

	\begin{minipage}{0.4\textwidth}
		\begin{flushleft} \large
			\emph{Authors:}\\
			% Ordre alphabetique sur les noms
			\textsc{Youcef Sekouri, William Fabre} 
		\end{flushleft}
	\end{minipage}
	~
	\begin{minipage}{0.4\textwidth}
		\begin{flushright} \large
			\emph{Teachers :} \\
			Mr \textsc{Matthieu Tuna},\textsc{Pirouz Bazarghan-Sabet}
		\end{flushright}
	\end{minipage}\\[2cm]

	%----------------------------------------------------------------------------------------
	%	DATE SECTION
	%----------------------------------------------------------------------------------------

	{\large Year 2019-2020}\\[2cm] % Date, change the \today to a set date if you want to be precise

\end{titlepage}

\newpage
\tableofcontents
\vspace*{3cm}
\begingroup\let\clearpage\relax

	\newpage
	\chapter{Introduction}

	%multilinecomments
	\begin{comment}
	\end{comment}

	% TODO quotation work use the biblio.bib to add references.
	TODOCHANGEHEREtest1234\cite{greenwade93}

	\newpage
	\chapter{Project Installation}
	% TODO tutoriel d'installation
	\newpage


	\newpage
	\bibliographystyle{IEEEtran}
	\bibliography{biblio}

	\newpage
	% \phantomsection
	\addcontentsline{toc}{chapter}{\listfigurename}
	\listoffigures



	\newpage
	\chapter{Annexe}
	\section{Glossary :}

\begin{itemize}
	\item EDA : Electronic design automation
		%integrated cells for ATPG, Boundary Scan...
	\item CTS : Clock Tree Synthesys : %TODO
	\item STA : Static Timing Analysis  %TODO operates independently of characterization reading both a Verilog netlist and multiple timing libraries in Liberty format 

	\item Formal Verification :
	\item DRC : Design Rule Check %TODO
	\item LVS : Logical Vs Schematic %TODO extraction of circuit
	\item IC : Integrated Circuit
	\item PCB : Printed Circuit Board
	\item Plot : %TODO
	\item Pad : %TODO
	\item FP : Floot Planing %TODO (guide the tool)
	\item ATPG : Automatic Test Pattern generation
	\item stuck-at fault : is a fault model for ATPG %TODO

	\item PVT : Process Voltage Temperature %TODO?
		%Lot to Lot
		%Wafer to Wafer
		%Dye to Dye
		%Corner-based analysis?

	% TODO
	\item Timing Constraints :
		Reg to Reg
		Input to Reg
		Reg to Output
		Input Delay
		Output Delay
		Input to output : Mealy path (Combinatory cannot be constraints) so you need to contraint input/output with absolute value


	\item Fanout : how much cells my signal can attack without being degraded

	\item Setup :
	\item Hold :

	\item .lib : LIBerty : Technology library source files containing all required information for synthesis and static timing analysis

	\item library max (WC=Worst Case) : The timing in this kind of library are the longest (Max path/Max Data path). They are used for setup analysis. 
	\item library min (BC=Best Case) : The timing in the kind of library are the shortest (Min path/Min Data path). They are used for hold analysis.
	

	\item WNS : Worst negative slack : The longest path in the design
	\item TNS : Total Negative Slack : The Sum of all the longest path in the design
	\item WLM : Wire-Load Model : Statistical Model that help having a coherent
		analysis of the circuit after synthesis (there is no wire yet).


	\item TA  : Timing Arc : Time since beginning of input evolution value and end of output establishing value
		* Cell
			Cell Delay : propagation delay inside a cell
			Fall Cell Delay : inside cell Delay when input transition from 1 to 0
			Rise Cell Delay : inside ell Delay when input transition from 0 to 1

		* Transition
			Transition Delay : Delay before establishing value on output ago the beginning of transition of output
				Fall Transition : output delay transition for an input which through from 1 to 0
				Rise Transition : output delay transition for an input which through from 1 to 0				

	\item Delay equation %TODO
	\item TC  : Timing Clean %TODO

	\item DC : Design Compiler
	\item DCP : Design Compiler Physical
	\item RC : RTL Compiler
	\item RCP : RTL Compiler Physical
	
	\item SS : Setup Slack : Real time between data latch arrival and clock latch edge, less setup timing. To be TC, all SS must be positif or void.
	\item HS :Hold Slack : Real time between end of data stablished and end of hold timing. To be TC, all HS must be positif or void.
	\item CDC : Clock Domain Crossing
	\item GDSII
	\item PnR : Place .\& Route
	\item 
	\item 
	\item 
	\item 
	\item 
	
	\item test at speed : %TODO
	\item timing clean :
	\item floor planning :
	\item power planning :
	\item  spare cells :
	\item  ICO : Ingeneer Change Order
	\item  Boundary Scan
	\item JTAG
	\item Trial Route
	\item  Timing Met
	\item Signoff
	
	\item LEF : Layer Exchange Format : library timing model, its a lite physical library for abstract model different from library used to GDSII
	\item SDC : Sinopsys Design Constraints
	\item Shmoo plots ("excel" table)
	\item 
	\item 
	\item 
	\item 
	\item 
	\item 
	\item 
	\item 
	\item 
	
\end{itemize}



\end{document}
